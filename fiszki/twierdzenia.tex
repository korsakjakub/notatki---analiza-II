\documentclass{article}
\usepackage[utf8]{inputenc}
\usepackage[margin=1.25in]{geometry}
\usepackage{mdframed}
\usepackage{subfiles}
\usepackage{subcaption}
\usepackage{graphicx}
\usepackage{wrapfig}
\usepackage{amsmath}
\usepackage{amsfonts}
\usepackage{polski}
\usepackage{import}
\usepackage{xifthen}
\usepackage{pdfpages}
\usepackage{transparent}

\newcommand{\incfig}[1]{%
    \def\svgwidth{\columnwidth}
    \IfFileExists{img/#1.pdf_tex}{
        \import{"img/"}{#1.pdf_tex}
    }
    {
        \import{"../img/"}{#1.pdf_tex}
    }
}
\newmdtheoremenv{tw}{Twierdzenie}
\newmdtheoremenv{stw}{Stwierdzenie}
\newtheorem{obserwacja}{Obserwacja}
\newtheorem{dowod}{Dowód}
\newmdtheoremenv{definicja}{Definicja}
\newtheorem{przyklad}{Przykład}
\newtheorem{pytanie}{Pytanie}
\newtheorem{uwaga}{Uwaga}
\begin{document}

\begin{tw}
Niech $f:U\subset \mathbb{R}^n \to \mathbb{R}^1, f\in \mathcal{C}^2(U), G: U_2\subset\mathbb{R}^n\to\mathbb{R}^m, G\in \mathcal{C}^2(U_2), \underset{x_0}{\exists} G(x_0) = 0, G'(x_0)$ - ma rząd maksymalny $(m)$ oraz \[
\underset{\Lambda}{\exists} \Lambda = \left[ \lambda_1,\ldots,\lambda_m \right], \lambda_i \in \mathbb{R}, f'(x_0) - \Lambda G'(x_0) = 0
.\] to jeżeli
\[
(f''(x_0)-\Lambda G''(x_0))(h_\parallel,h_\parallel) > 0,
h_\parallel \overset{\text{def}}{=} \left \{G'(x_0) h_\parallel = 0 \right \}
.\] to $f$ posiada w $x_0$ minimum lokalne $(<0$, to maksimum lokalne$)$ na zbiorze $M = \left\{ x\in \mathbb{R}^n, G(x) = 0 \right\} $
\end{tw}
\begin{tw}
Niech $[a,b] \subset \mathbb{R}, \mathcal{O}\subset\mathbb{R}^n, \mathcal{O}$ - domknięty i $f:[a,b]\times\mathcal{O}\to\mathcal{O}$ takie, że $f$ - ciągła na $[a,b]\times\mathcal{O}$ oraz $f$ spełnia warunek Lipschitza na $\mathcal{O}$, to znaczy:
\[
\underset{L>0}{\exists}. \underset{t\in[a,b]}{\forall}. \underset{x,x'\in\mathcal{O}}{\forall} \Vert f(t,x) - f(t,x') \Vert \leq L \Vert x-x' \Vert
.\] Wówczas \[
\underset{t_0\in[a,b]}{\forall} . \underset{x_0\in\mathcal{O}}{\forall} . \underset{\varepsilon>0}{\exists} \text{, że dla } t\in ]t_0-\varepsilon, t_0+\varepsilon[
\]
równanie ma jednoznaczne rozwiązania, które są ciągłe ze względu na $x_0$
\begin{equation}
\begin{cases}\label{eq:eq21}
\frac{dx}{dt} = f(t,x)\\
x(t_0) = x_0
\end{cases}
\end{equation}
\end{tw}
\begin{tw}
Jeżeli odwzorowania
\begin{align*}
&t\in [a,b]\to A(t)\\
&t\in [a,b]\to b(t)
.\end{align*}
Gdzie $A(t)\in L(x,x), b(t) : \mathbb{R}^1\to X$ są ciągłe, to równanie
\[
\frac{d}{dt}x(t) = A(t)x(t) + b(t),\quad x(t_0) = x_0
.\]
Ma dla dowolnych $t_0\in[a,b], x_0\in X$ jednoznacznie określone rozwiązanie na $t\in]a,b[$\\
Czym to się różni od twierdzenia o jednoznaczności warunku Cauchy? Nie ma tutaj mowy o żadnej lipszycowalności. Zawężono za to klasę funkcji występującej w równaniu. Zamiast $]t_0-\varepsilon,t_0+\varepsilon[\times \mathcal{O}$, mamy $]a,b[ \times X$
\end{tw}
\begin{tw}
(Liouville)\\
Jeżeli $R(t,t_0)$ - rezolwenta dla problemu
\begin{align*}
&\frac{dx}{dt} = A(x)x(t)\\
&x(t_0) = x_0
.\end{align*}
i $x\in \mathbb{R}^n$, to $w(t) = w(t_0)e^{\int_{t_0}^t tr(A(s))ds}$, gdzie $w(t) = \det R(t,t_0)$ i $w(t)$ nazywamy wrońskianem.\\
\end{tw}
\begin{tw}
(Lebesque) Niech $P$ - zbiór nieciągłości funkcji $f: D\to\mathbb{R}$, $f$ - ograniczona na $D$, $D$ - \ldots jest zbiorem miary Lebesque'a zera  $\iff$ $f$ - całkowalna na $D$.
\end{tw}
\begin{stw}
($X$ - domknięty,ograniczony) $\iff$ ($X$-zbiór zwarty)
\end{stw}
\begin{tw}
(Lebesgue'a) niech $D$ - kostka, $D\subset \mathbb{R}^n$, $f: D\to \mathbb{R}$, $f$ - ograniczona.\\
Wówczas $f$ - (całkowalna na $D$ ) $\iff$ (zbiór nieciągłości funkcji $f$ jest miary Lebesgue'a zero)
\end{tw}
\begin{tw}
(Fubiniego)\\
Niech $f: A\times B\to \mathbb{R}$. $A\subset\mathbb{R}^l, B\subset\mathbb{R}^k, A\times B\subset\mathbb{R}^n$, $f$ - ograniczona i całkowalna na $A\times B$. Oznaczmy $x^l\in A, y^k\in B$, $A,B$ - kostki.\\
Niech \[
\varphi(x) = \overline{\int_B}f(x^l,y^k)dy^k, \psi(x) = \underline{\int_B} f(x^l, y^k)dy^k
.\]
Wówczas \[
\int_{A\times B} f = \int_A \varphi = \int_A \psi
.\]
\begin{uwaga}
całkowalnośc na $A\times B$ nie oznacza całkowalności na np. $B$.
\end{uwaga}
\end{tw}
\begin{tw}
(O zamianie zmiennych)\\
Niech $\Theta, \Omega$ - zbiory otwarte w $\mathbb{R}^n$ i $\xi: \Omega\to \Theta$, $f: \Theta\to \mathbb{R}$, $f$ - ograniczona i całkowalna. $\xi$ - klasy $\mathcal{C}^1$ na $\Omega$, $\xi^{-1}$ klasy $\mathcal{C}^{1}$ na $\Theta$. Wtedy
\begin{equation}
\int_{\Theta} f(x) dx =  \int_{\Omega} f(\xi(t)) | \det \xi'(t) | dt.
\end{equation}
$x=(x^1,\ldots,x^n)\in \Theta, t=(t^1,\ldots,t^n)\in \Omega$
\end{tw}
\begin{tw}
Jeżeli $f$ - różniczkowalna w $x_0 \in U$, to dla dowolnego $e\in U$, $$\nabla_e f(x_0) = f'(x_0)e$$
\end{tw}
\begin{tw}
Niech $O\subset\mathbb{R}^{n}, O$ - otwarty. $f: O\to Y, x_0\in O$.

Jeżeli istnieją pochodne cząstkowe $\frac{\partial}{\partial x_i} f, i=1,\dots,n$ i są ciągłe w $x_0$, wtedy $\underset{h\in\mathbb{R}^n}{\forall} f(x_0+h)-f(x_0)=\sum_{i=1}^{n} \frac{\partial f}{\partial x_i} h^i+r(x_0,h)$, gdzie $\frac{r(x_0,h)}{||h||}\to0$

\end{tw}
\pagebreak
\begin{tw}
Niech $G:U \to Y, U\subset X, U \text{ - otwarte }, X$ - przestrzeń wektorowa unormowana, $F: G(U) \to Z, G(U) \subset V$\\
$G$ - różniczkowalna w $x_0\in U$, $F$ - różniczkowalna w $G(x_0)\in U$.\\
$$G(x_0 + h_1) - G(x_0) = G'(x_0)h_1+r_1(x_0,h_1)\text{, gdy }\frac{r(x_0,h_1)}{||h_1||_x} \to 0$$
$$F(y_0 + h_2) - F(y_0) = F'(y_0)h_2+r_2(y_0,h_2)\text{, gdy }\frac{r(y_0,h_2)}{||h_2||_y} \to 0$$
$$\text{Wówczas: } (F \circ G ) \text{ - różniczkowalna w } x_0$$
$$\text{oraz } (F \circ G)' (x_0) = \left . F'(x)\right |_{x=G(x_0)} G'(x_0)$$
\end{tw}
\begin{tw}
Niech $f: \mathcal{O}\to\mathbb{R}, \mathcal{O}\subset \mathbb{R}^n$, otwarty i $f\in\mathcal{C}^2(\mathcal{O})$, wówczas
$$\frac{\partial^2 f}{\partial x^i \partial x^j} = \frac{\partial^2 f}{\partial x^j \partial x^i}; i,j=1,\dots,n$$
\end{tw}
\begin{stw}
jeżeli $f: \mathcal{O} \rightarrow \mathbb{R}, \mathcal{O}$ - otwarty, $x_0 \in \mathcal{O}, f$ - posiada w $x_0$ minimum lub maksimum lokalne, to $$\frac{\partial f}{\partial x^i} (x_0) = 0, i = 1,\dots,n$$
(działa tylko w prawo, bo możliwe punkty przegięcia ((siodła)) )
\end{stw}
\begin{tw}
Niech $f: \mathcal{O} \to \mathbb{R}, \quad\mathcal{O}\subset\mathbb{R}^n, \quad x_0\in\mathcal{O}, \quad \mathcal{O} \text{ - otwarty, a } f \text{ - klasy } C^{2p} (\mathcal{O})$ oraz $f'(x_0) = 0, f''(x_0) = 0,\dots,f^{(2p-1)} (x_0) = 0$
i $$\underset{c > 0}{\exists} \quad\underset{\eta > 0}{\exists} \quad\underset{h\in K(x_0,\eta)}{\forall}: \quad \sum_{\substack{i_1 = 1\\ \vdots \\ i_{2p} = 1}}^n \frac{\partial^{(2p)} f}{\partial x^{i_1} \dots \partial x^{i_{2p}}} (x_0) h^{i_1} \dots h^{i_{2p}} \geq c ||h||^{2p} (\leq c||h||^{2p})$$
to $f$ ma w $x_0$ minimum (maksimum) lokalne.
\end{tw}
\begin{tw}
Twierdzenie ($L$ - ograniczone) $\iff$ ($L$ - ciągłe)
\end{tw}
\begin{stw}
Niech $f: U\subset \mathbb{R}^m \to \mathbb{R}^n, U$ - otwarte, wypukły
$\underset{M}{\exists}.\underset{x\in U}{\forall}||f'(x)||\leq M$, to $\underset{a,b\in U}{\forall}||f(b)-f(a)||_n \leq M||b-a||_m$ \begin{tiny}(jakiekolwiek skojarzenia z Twierdzeniem Lagrange zupełnie przypadkowe *wink* *wink*)\end{tiny}
\end{stw}
\begin{tw}
Jeżeli ciąg $\{x_0, P(x_0), \dots \} $ - zbieżny i $P$ - ciągłe, to jest on zbieżny do punktu stałego.
\end{tw}
\begin{tw}
(Zasada Banacha o lustrach)\\
Jeżeli $P: X \to X, P$ - zwężające, to
\begin{align}\label{eq:banach}
&\text{1. } \underset{x_0 \in X}{\forall}\quad \{x_0,P(x_0),P(P(x_0)),\dots\}) \text{ - Zbieżny do punktu stałego } \tilde x\\
&\text{2. Istnieje tylko jedno }\tilde x\\
&\text{3. } \underset{m}{\forall}\quad d(x_m,\tilde x) < \frac{q^m}{1-q} d(x_1, x_0)
\end{align}
\end{tw}
\begin{tw}
(o lokalnej odwracalności)\\

Niech $f: E \to E, E$ - otwarty, $E\subset \mathbb{R}^N, f$ - różniczkowalna w sposób ciągły na $E$.

$(f \text{ - klasy } \mathcal{C}^1 (E)), \underset{a,b\in E}{\exists}:f(a) = b \land f'(a) \text{ - odwracalna } (det(f'(a))\neq 0)$, to:
\begin{align*}
&1. \underset{U,V\subset E}{\exists}, \underset{a\in U, b\in V}{\exists}, U,V \text{ - otwarte, }f \text{ - bijekcja między } U,V\\
&2. \underset{g: V\to U}{\exists}.\underset{x\in V}{\forall}, f(g(x)) = x, g\text{ - ciągła i różniczkowalna na } V\\
\end{align*}
\end{tw}
\begin{tw}
(o funkcji uwikłanej)\\
Niech $H:E\subset\mathbb{R}^{n+m}\to\mathbb{R}^{m},H\in\mathcal{C}^{1}$ na $E$. $(x_0,y_0)\in E, H(x_0,y_0)=0, (x_0,y_0) = (x_0^1,\ldots,x_0^n,y_0^1,\ldots,y_0^m), H$ - odwracalna.\\
Wówczas istnieje  $U\subset E$ takie, że $(x_0,y_0)\in U, \underset{W\subset \mathbb{R}^{n}}{\exists} $, że $x_0\in W, \underset{x\in W}{\forall} \underset{y}{\exists !} H(x,y) = 0, (x,y) \in U$.\\
Jeżeli $y= \varphi(x)$, to $\varphi:\mathbb{R}^{n}\to\mathbb{R}^{m}$ i $\varphi\in \mathcal{C}^{1}$ na $W$. $\varphi'(x) = -(H_y')^{-1}H_x'$
\end{tw}
\end{document}
