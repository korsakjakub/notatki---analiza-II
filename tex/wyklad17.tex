\documentclass[../main.tex]{subfiles}
\graphicspath{
    {"../img/"}
    {"img/"}
}

\begin{document}
\begin{pytanie}
    Jak pokazać, że zbiór Cantora jest niepusty?
\end{pytanie}
\begin{stw}
    Przeliczalna ilość zbiorów miary Lebesgue'a zero jest też zbiorem miary Lebesgue'a zero.
\end{stw}
\begin{proof}
    Weźmy rodzinę zbioru $X_n \in \mathbb{R}^n, n\in \mathbb{N}_+$, $X_n$ - zbiór miary Lebesgue'a zero.\\
    Weźmy rodzinę kostek $P_n$, gdzie $P_i$ - kostka z $\mathbb{R}^n$ taka, że
    \[
    \left| P_i \right| < \frac{\varepsilon}{2^i}
    .\]
(możemy tak zrobić, bo $X$ - miary Lebesgue'a zero)\\
wówczas $X\subset P = P_1 \cup P_2 \cup \ldots$
\[
\left| P \right| = \left| P_1 \cup P_2 \cup \ldots  \right| \le \sum_{i = 1}^{\infty} \frac{\varepsilon}{2^i} = \varepsilon\cdot \frac{1}{2}\cdot \frac{1}{1-\frac{1}{2}} = \varepsilon
.\]
\end{proof}
\begin{pytanie}
    Jak pociąć prostokąt?
\end{pytanie}
\textit{będzie trzeba wprowadzić język.}\\
weźmy kostkę z $\mathbb{R}^n$. Wtedy
\[
    P = \left[ a_1, b_1 \right] \times \ldots \times \left[ a_n, b_n \right]
,\]
niech $\Pi_i$ - podział przedziału $\left[ a_i, b_i \right]$, $i = 1, \ldots, n$.\\
Zatem podział kostki $P$ wyznaczy kolekcja podziałów
\[
    \Pi = \left\{ \Pi_1, \Pi_2, \ldots, \Pi_n \right\}
.\]
W takim razie $P$ możemy przedstawić jako sumę
\[
    P = \bigcup_{i = 1}^k Q_i \text{, gdzie } Q_i = \left[ a_1^{i_1}, b_1^{i_1} \right] \times \ldots \times \left[ a_n^{i_n}, b_n^{i_n} \right]
.\]
$\left[ a_j^{i_j}, b_j^{i_j} \right] $ - jeden z elementów podziału odcinka $[a_1, b_1]$ przy podziale $\Pi_1, \ldots, $ itp.\\
Wówczas zauważmy, że jeżeli zdefiniujemy
\[
    int(Q_i) = \left] a_1^{i_1}, b_1^{i_1}\right[ \times \ldots
,\]
to wtedy
\[
    int(Q_i) \land int(Q_j) = \phi
.\]
Poza tym
\[
\left| Q_i \right| \overset{\text{def}}{=} \left| b_1^{i_1} - a_1^{i_1} \right| \cdot \ldots \cdot \left| b_n^{i_n} - a_n^{i_n} \right|
.\]
W związku z tym
\[
\left| P \right| = \sum_i \left| Q_i \right|
.\]
\textbf{Uwaga:} czasami zamiast pisać $\Pi = \left\{ \Pi_1, \ldots \right\} $, piszemy $\Pi = \left\{ Q_1, \ldots \right\} $

\begin{definicja}
    Rozważmy dwa podziały:
    \begin{align*}
        \Pi_1 &= \left\{ Q_1, \ldots, Q_k \right\} \\
        \Pi_2 &= \left\{ R_1, \ldots, R_s \right\}
    .\end{align*}
    Podział $\Pi_2$ nazywamy drobniejszym, jeżeli dla $\Pi_1 \neq \Pi_2$
    \[
        \underset{a_j}{\forall}, j\in \left\{ 1, \ldots, k \right\}\quad \underset{i\in \left\{ R_{m_1}, \ldots, R_{m_j} \right\} }{\exists} , Q_j = R_{m_1}\cup \ldots \cup R_{m_j}
    .\]
\end{definicja}
\begin{definicja}
    Suma górna dla $f: P\subset \mathbb{R}^n \to \mathbb{R}$
     \[
         \overline{S}(f, \Pi) = \sum_{Q_i\in \Pi} \sup_{x\in Q_i} f(x) \cdot \left| Q_i \right|
     ,\]
 oraz suma dolna
 \[
     \underline{S}(f, \Pi) = \sum_{Q_i\in \Pi} \inf_{x\in Q_i} f(x) \cdot \left| Q_i \right|
 .\]
\end{definicja}
\begin{definicja}
    Całka górna:
    \[
        \overline{\int_{p}} f = \inf_{\Pi} \overline{S}(f, \Pi)
    ,\]
oraz
\[
    \underline{\int_{p}} f = \sup_{\Pi} \underline{S}(f, \Pi)
.\]
\end{definicja}
\begin{obserwacja}
   Jeżeli $\Pi_2$ - podział drobniejszy, niż $\Pi_1$, to
    \[
        \underline{S}(f, \Pi_1) \le \underline{S}(f, \Pi_2) \le \overline{S}(f, \Pi_2) \le \overline{S}(f, \Pi_1)
    .\]
\end{obserwacja}
\begin{pytanie}
    Czym to się właściwie różni od całki na $\mathbb{R}$?
\end{pytanie}
\begin{pytanie}
    Czy całkę na np. $\mathbb{R}^2$ możemy policzyć przy pomocy dwóch całek na $\mathbb{R}$?
\end{pytanie}
\begin{przyklad}
 \[
     p = \left[ 0,1 \right] \times \left[ 0,2 \right], \quad f(x,y) = xy^2
.\]
\[
    \int_{p} xy^2 \overset{\text{??}}{=}  \int_{0}^1 dx \int_0^2 dy\cdot  xy^2 \overset{\text{??}}{=} \int_0^2 dy \int_0^1 dx \cdot xy^2
.\]
\[
    \int_0^1dx\left[ \frac{xy^3}{3} \right]^2_0 = \left[ \frac{x^2\cdot 2^3}{2\cdot 3} \right]^1_0 = \frac{4}{3}
.\]
\[
    \int_0^2 dy \left[ \frac{x^2y^2}{2} \right]_0^1 = \left[\frac{y^3}{2\cdot 3}\right]_0^2 = \frac{4}{3}
.\]
\end{przyklad}
\begin{pytanie}
    Co robimy ze zbiorami, które nie są prostokątami?
\end{pytanie}
\begin{przyklad}
    niech $A\subset P\subset \mathbb{R}^n$, wprowadźmy funkcję
    \[
        X_A(x) = \begin{cases}
            1 &x\in A\\
            0 &x\not\in A
        \end{cases}
    ,\]
wówczas jeżeli
\[
f: P\to\mathbb{R}, A\subset P
,\]
to
\[
\int_{A}f \overset{\text{def}}{=} \int_{P} f \cdot X_A
.\]
Jeżeli $f$ - takie, że $f: A\to\mathbb{R}$, to definiujemy funkcję
\[
    \tilde f(x) = \begin{cases}
        f &x\in A\\
        0 &x\not\in A
    \end{cases}
\]
i wtedy
\[
    \int_{A} f = \int_{P} \tilde f
.\]
\end{przyklad}

Konsekwencją takiego podejścia jest konieczność radzenia sobie z całkami z funkcji nieciągłych. Oznacza to, że warunek na całkowalność punktu musi być związany z nieciągłością. W tym celu wprowadzamy kilka nowych pojęć:
\begin{definicja}
    Wahanie funkcji:\\
    niech $f: A\subset \mathbb{R}^n\to \mathbb{R}$, niech $x_0\in \int(A)$. Wahaniem funkcji w punkcie $x_0$ nazywamy wielkość
    \[
        O(f,x_0) \overset{\text{def}}{=} \lim_{r\to 0} \left| \sup_{K(x_0, r)}f(x) - \inf_{K(x_0, r)}f(x) \right|
    .\]
\end{definicja}
\begin{stw}
    Niech $A$ - domknięty, $A\subset \mathbb{R}^n$, $f: A\to \mathbb{R}$.\\
    \[
        A_\varepsilon = \left\{ x\in A: O(f, x) \ge \varepsilon \right\}
    ,\]
wówczas $A_\varepsilon$ teże jest zbiorem domkniętym.
\end{stw}
\begin{proof}
    Pokażemy, że zbiór $A_\varepsilon'$ jest zbiorem otwartym.\\
    Mamy dwa przypadki:
    \begin{enumerate}
        \item $x\in A'_\varepsilon$, $x\not\in A$, czyli $x\in A'$ więc $\underset{r}{\exists} K(x,r)\in A'$ (bo $A'$ jest otwarty)
        \item $x\in A$, $x\in A'_\varepsilon$, czyli $O(f,x) < \varepsilon$
    \end{enumerate}
    Chcemy pokazać, że
    \[
        \underset{r>0}{\exists} \quad \underset{y\in K(x,r)}{\forall} O(f,y) < \varepsilon
    .\]
czyli znajdziemy takie otoczenie $x\in A'_\varepsilon$, że wszystkie elementy z tego otoczenia też należą do $A'_\varepsilon$ czyli $A'_\varepsilon$ jest otwarty.\\
Wiemy, że
\[
    \lim_{r\to 0} \left| \sup_{x'\in K(x,r)}f(x') - \inf_{x'\in K(x,r)}f(x') \right| = c < \varepsilon
.\]
Z definicji granicy oraz warunku wyżej wiemy, że
\[
    \underset{r}{\exists} \left| \sup_{x'\in K(x,r)}f(x) - \inf_{x'\in K(x,r)}f(x) \right| < \varepsilon
,\]
zatem dla dowolnego $y\in K(x,r)$ mamy
 \[
     \underset{r'}{\exists} r' = r- \left\Vert x - y \right\Vert: \underset{y'\in K(y,r)}{\forall} \left| \sup f(y') - \inf f(y') \right| < \varepsilon
,\]
czyli
\[
    O(f,y') < \varepsilon \to \underset{y'}{\forall} y'\in K(y, r')\subset K(x,r)
,\]
co oznacza, że punkt $x$ jest punktem wewnętrznym $A'_\varepsilon$, czyli $A'_\varepsilon$ jest otwarty, więc $A_\varepsilon$ - domknięty.
\end{proof}
\end{document}
