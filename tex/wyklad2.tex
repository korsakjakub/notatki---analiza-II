\documentclass[../main.tex]{subfiles}
\begin{document}

\begin{definicja}
Norma\\
    Niech $X$ - przestrzeń wektorowa.\\
Odwzorowanie $||.||: \mathbb{X}\to \mathbb{R}$ nazywamy normą, jeżeli:

\begin{align}
    \underset{x\in X}{\forall} \quad &||x|| \geq 0\\
    \underset{\alpha\in\mathbb{R}}{\forall}, \underset{x\in \mathbb{X}}{\forall} \quad &|| \alpha x|| = |\alpha| ||x||\\
    \underset{x,y \in X}{\forall} \quad &||x+y|| \leq ||x|| + ||y||\\
    \underset{x\in X}{\forall} \quad &||x|| = 0 \iff x = 0
\end{align}

Przestrzeń $X$ wraz z normą $||.||$ nazywamy przestrzenią unormowaną (\textit{spoiler:} przestrzenią Banacha).
\end{definicja}

\begin{przyklad}
    Przykładowa norma:\\
    \[
        ||v|| = \sqrt{(x)^2 + (y)^2}, X = \mathbb{R}^{n}
    .\]
    \[
        X \ni v \implies||v|| = sup(|x^1|,\dots).
    \]
    Jeżeli $f\in \mathcal{C}([a,b])$,
    to norma wygląda tak:
    \[
        ||f|| = sup_{x\in{[a,b]}} (f(x))
    .\]
\end{przyklad}
\begin{przyklad}
    \[
        \mathbb{R}^2_2 \ni \begin{bmatrix} a&b\\c&d \end{bmatrix} = v
    \]
    \[
        \Vert v \Vert = \max \left\{ |a|, |b|, |c|, |d| \right\}
    .\]
\end{przyklad}

    \textbf{Uwaga:} mając normę możemy zdefiniować metrykę $\underset{x,y\in X}{\forall} d(x,y) = ||x-y||$, natomiast nie każdą metrykę da się utworzyć przy pomocy normy.

\begin{przyklad}
    metryka zdefiniowana przy pomocy normy ma np. taką własność:
    \[
        d(ax,ay) = ||ax - ay|| = |a| ||x-y|| = a d(x,y),
    \]
    czyli taka metryka się skaluje natomiast funkcja
    \[ d(x,y) =
    \begin{cases}
        1 & \quad x\neq y\\
        0 & \quad x=y\\
    \end{cases}
    \]
    jest metryką, ale tej własności nie posiada.
\end{przyklad}


\begin{definicja}
    Pochodna mocna (trzecie podejście)
\[
 \lim\limits_{h \to 0}\frac{f(x+h) - f(x)}{h} = f'(x_0), \text{ dla }x\in V\subset \mathbb{R}^{n}
 .\]
 - taka definicja jest niemożliwa (nie mamy dzielenia wektorów).
\end{definicja}


\[
    f(x+h)-f(x)=f'(x_0) h + r(x_0,h), \text{ gdzie } \frac{r(x_0,h)}{||h||}\to 0 \text{ przy } ||h||\to 0
\]
ale to może mieć już inną dziedzinę

\begin{definicja}

    Niech $U \subset X, V\subset Y$\\
    $U,V\text{ - otwarte, }\quad T:U\to V\\
    x,h\in U$

    Mówimy, że $T$ - różniczkowalne w punkcie $x_0$, jeżeli prawdziwy jest wzór
    \[
        \underset{h\in U}{\forall} \quad T(x_0+h) - T(x_0) = L_{x_0} (h) + r(x_0,h),
    \]
    gdzie $\frac{r(x_0,h)}{||h||}\to 0$, a $L_{x_0}$ - liniowe $: X\to Y$.
\end{definicja}


\vspace{0.3cm}
Odwzorowanie $L_{x_0} (h)$ nazywamy pochodną T w punkcie $x_0$.
Czasami $L_{x_0}(h)$ możemy przedstawić w postaci $L_{x_0} (h) = T'(x_0) h$, to $T'(x_0)$ nazywamy pochodną odwzorowania T.

\vspace{0.3cm}
\textbf{Uwaga:} Dlaczego $L_{x_0}(h)$, a nie $T'(x_0) h$?
\vspace{0.3cm}

Dlatego, że czasami pochodna może wyglądać tak:

\[
    \int_0^1 h(x)\sin{x}dx
,\]
a tego nie da sie przedstawić jako
\[
    \left ( \int_0^1 \sin{x}dx \right ) h(x).
\]

\begin{przyklad}


   $T(x+h) - T(x) = T'(x_0)h+r(x_0,h)$
    \begin{align}
        &1. T: \mathbb{R}\to \mathbb{R}^{3}, \text{ czyli } x_0\in \mathbb{R}, h\in R \implies T(x) = \begin{bmatrix} -\\-\\- \end{bmatrix}  T'(x) = \begin{bmatrix} -\\-\\- \end{bmatrix} \\
            &2. T:\mathbb{R}^3 \to \mathbb{R} \quad x_0 = \begin{bmatrix} -\\-\\- \end{bmatrix}  h = \begin{bmatrix} -\\-\\- \end{bmatrix} , T'(x) = \left[ -,-,- \right] \\
                &3. T:\mathbb{R}^2 \to \mathbb{R}^3 \quad x_0 \begin{bmatrix} -\\- \end{bmatrix}  h = \begin{bmatrix} -\\- \end{bmatrix}, T(x) = \begin{bmatrix} -\\-\\- \end{bmatrix} , T'(x) = \begin{bmatrix} -&-&-\\-&-&- \end{bmatrix} \\
    .\end{align}
\end{przyklad}
\begin{przyklad}
    \[
        f(x,y) = xy^2, h=\binom{h_x}{h_y}.
    \]
       \begin{align*}
           &f(x_0+h_x,y_0+h_y) - f(x_0,y_0) = \\
           &= (x_0+hx)(y_0+hy)^2 - x_0y_0^2 = \\
           &= x_0y_0^2 + 2y_0x_0h_y + x_0h_y^2 + h_yy_0^2 + h_xh_y 2y_0 + h_xh_y = \\
           &= \left[ y_0^2, 2x\cdot x_0 \right] \begin{bmatrix}h_x\\h_y  \end{bmatrix} + x_0h_y^2+h_xh_y^2+2y_0h_xh_y
       .\end{align*}
\end{przyklad}

\pagebreak
\begin{pytanie}
    Czy $\frac{r(x_0,h)}{||h||}\underset{h \to 0}{\longrightarrow} 0$?
\end{pytanie}
Weźmy $\left\Vert \begin{bmatrix} h_x\\h_y \end{bmatrix} \right\Vert = \sup\{|h_x|,|h_y|\}$, wówczas\\
\[
    x_0h_y^2 + h_xh_y^2 + 2y_0h_xh_y \leq x_0 ||h||^2 + ||h||^3 + 2y_0||h||^2 = ||h||^2(x_0 +2y_0 + ||h||),
\]\\
zatem
\[
    \frac{r(x_0,h)}{||h||} \leq \frac{||h||^2(|x_0|+2y_0+||h||)}{||h||} \to 0.
\]

\[
    f(x,y)=xy^2, T'(x)=[y^2,2xy].
\]
zauważmy, że
\[
    y^2= \frac{\partial}{\partial x} f, 2xy = \frac{\partial}{\partial y} f.
\]
\textbf{Uwaga: }skąd wiemy, że gdy $h\to 0$, to $||h||\to 0$?\\
Czyli: czy norma jest odwzorowaniem ciągłym w $h=0$?

\textit{odpowiedź za tydzień}

\begin{tw}
Jeżeli $f$ - różniczkowalna w $x_0 \in U$, to dla dowolnego $e\in U$, $$\nabla_e f(x_0) = f'(x_0)e$$
\end{tw}

\begin{dowod}
    skoro $f$ - różniczkowalna, to
    \begin{equation}
        \label{eq: eq_2.1}
        \underset{h\in U}{\forall} f(x_0+h) - f(x_0) = f'(x_0)h + r(x_0,h), \frac{r(x,h)}{\Vert h \Vert } \underset{\Vert h \Vert \to 0}{\longrightarrow} 0
    \end{equation}

    \[
        \underset{h_x, h_y}{\forall} \frac{\sqrt{h_x\cdot h_y} }{\left\Vert h \right\Vert } \underset{h\to 0}{\longrightarrow} 0
    .\]
Niech $\left\Vert h \right\Vert = \sup \left\{ |h_x|, |h_y| \right\}, |h_x| > |h_y| \implies \left\Vert h \right\Vert = |h_x|$
\[
    \frac{\sqrt{|h_x \cdot h_y|} }{\left\Vert h \right\Vert } = \frac{\sqrt{|h_x \cdot h_y}}{h_x} \not \to  0
.\]


\end{dowod}


\begin{pytanie}
Czy z faktu istnienia pochodnych cząstkowych wynika różniczkowalność funkcji?
\end{pytanie}

\begin{przyklad}
\end{przyklad}
$f(x,y) = \sqrt{|xy|}, x_0=\binom{0}{0}$, dla $f(x,y)$ policzyliśmy pochodne cząstkowe w $x_0 \quad \frac{\partial}{\partial x} f = 0, \frac{\partial}{\partial y} f = 0$.
\vspace{0.3cm}

$h=\binom{h_x}{h_y}, x_0=\binom{0}{0}, f(x_0+h)-f(x_0) = \sqrt{h_xh_y} - \sqrt{0} = \sqrt{h_xh_y} = (0,0)\binom{h_x}{h_y} + \sqrt{h_xh_y}$, gdzie $r(x_0,h) = \sqrt{h_xh_y}$.\\
Czyli $f$ - różniczkowalna, jeżeli $\underset{h_x,h_y}{\forall}\quad \frac{\sqrt{h_xh_y}}{||h||}\to0$.\\
\vspace{0.3cm}
Niech $||h|| = \sup\{|h_x|,|h_y|\}$ i niech $|h_x|>|h_y|$. $||h|| = |h_x|.$\\
Dalej mamy: $\frac{\sqrt{h_xh_y}}{|h_x|}\sqrt{\frac{h_y}{h_x}} \not\to 0 \text{ przy }h_x\to0$, $\sqrt{\frac{|h_y|}{|h_x|}}=\sqrt{\frac{1}{2}}$
\vspace{0.3cm}

\textbf{Czyli istnienie pochodnych cząstkowych nie oznacza różniczkowalności.}

\begin{tw}
Niech $O\subset\mathbb{R}^{n}, O$ - otwarty. $f: O\to Y, x_0\in O$.

Jeżeli istnieją pochodne cząstkowe $\frac{\partial}{\partial x_i} f, i=1,\dots,n$ i są ciągłe w $x_0$, wtedy
    \[
        \underset{h\in\mathbb{R}^n}{\forall} f(x_0+h)-f(x_0)=\sum_{i=1}^{n} \frac{\partial f}{\partial x_i} h^i+r(x_0,h)
    ,\]
    gdzie $\frac{r(x_0,h)}{||h||}\to0$
\end{tw}

\begin{proof}
    (dla $O=\mathbb{R}^3$)

    Niech $x_0 = \left [ \begin{matrix}
        x_0^1\\
        x_0^2\\
        x_0^3
    \end{matrix}
    \right ], h = \left [ \begin{matrix}
        h^1\\
        h^2\\
        h^3
    \end{matrix} \right ]$

    \begin{align*}
        &f(x_0^1+h^1,x_0^2+h_2,x_0^3+h^3)-f(x_0^1,x_0^2,x_0^3)=\\
        &=f(x_0^1+h^1,x_0^2+h^2,x_0^3+h^3)-f(x_0^1+h^1,x_0^2+h^2,x_0^3)+\\
        &+f(x_0^1+h^1,x_0^2+h^2,x_0^3)-f(x_0^1+h^1,x_0^2,x_0^3)+\\
        &+f(x_0^1+h^1,x_0^2,x_0^3)-f(x_0^1,x_0^2,x_0^3) \underset{\text{tw. o w. średniej}}{=}\\
        &\frac{\partial}{\partial x_0^1} f (c_1)h^1 + \frac{\partial}{\partial x_0^2} f (x_0^1+h^1,c_2,x_0^3)h^2+\frac{\partial}{\partial x_0^3} f (x_0^1+h^1,x_0^2+h_2,c_3)h^3=\\
        &(\frac{\partial f}{\partial x^1} (c_1,x_0^2,x_0^3) - \frac{\partial f}{\partial x^1} (x_0^1,x_0^2,x_0^3))h^1 +\\
        &+ (\frac{\partial f}{\partial x^2} (x_0^1 + h^1, c_2, x_0^3) - \frac{\partial f}{\partial x^2} (x_0^1, x_0^2, x_0^3))h^2 +\\
        &+ (\frac{\partial f}{\partial x^3} (x_0^1+h^1,x_0^2+h^2,c_3) - \frac{\partial f}{\partial x^3} (x_0^1, x_0^2, x_0^3))h^3\\
    .\end{align*}

    gdzie $c_1\in ]x_0^1,x_0^1+h^1[,\quad c_2\in ]x_0^2,x_0^2+h^2[,\quad c_3\in ]x_0^3,x_0^3+h^3[$

    Wystarczy pokazać, że $\frac{r(x_0,h)}{||h||}\to 0$, gdy $h\to 0$.

    Zauważmy, że każde wyrażenie tworzące resztę jest postaci \textit{coś} $h^i$, a $\lim\limits_{||h|| \to 0}\frac{h^i}{||h||} =$ \textit{dla normy np.} $||h|| = max{|h^i|}\neq 0$.
    (np. $\frac{h^1}{h^1} \to 1$)

    Oznacza to, że jeżeli $\frac{r(x,h)}{||h||}\to 0$ - spełniony, to każde wyrażenie typu
    $$ \Big ( \frac{\partial f}{\partial x^1} (c_1,x_0^2,x_0^3) - \frac{\partial f}{\partial x^1} (x_0^1,x_0^2,x_0^3)\Big ) h^1 \to 0$$

    Czyli np. $\lim\limits_{||h|| \to 0}\frac{\partial f}{\partial x^1} (c_1,x_0^2,x_0^3) = \frac{\partial f}{\partial x^1} (x_0^1,x_0^2,x_0^3) \iff (\frac{\partial f}{\partial x^1}  \text{ - ciągła} )$
\end{proof}
\end{document}
