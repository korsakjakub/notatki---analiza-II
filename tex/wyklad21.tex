\documentclass[../main.tex]{subfiles}
\graphicspath{
    {"../img/"}
    {"img/"}
}

\begin{document}
    Chcemy powiedzieć co to są wektory w takim świecie? Zaczniemy rysować krzywą po powierzchni.
    \begin{figure}[h]
        \centering
        \incfig{fig_51}
        \caption{Wymiar pączka może być większy! $m>n$}
        \label{fig:}
    \end{figure}

    Niech $M$ - rozmaitość. Odwzorowanie $\sigma: ]-\varepsilon,\varepsilon[ \subset\mathbb{R} \to \sigma(t)\in M$ nazywamy krzywą na $M$. $\sigma$ jest klasy $\mathcal{C}^\infty$
     \begin{przyklad}
         (spirala na walcu)
         \[
             \sigma: ]-\varepsilon,\varepsilon[ \to \begin{bmatrix} \cos(t)\\ \sin(t) \\ t \end{bmatrix}
         .\]
    \end{przyklad}
    \begin{definicja}
        Niech $p\in M$, $\sigma_1,\sigma_2$ - krzywe na $M$ takie, że $\sigma_1(0) = \sigma_2(0) = P$. Mówimy, że $\sigma_1$ i $\sigma_2$ są \underline{styczne w punkcie $P$}, jeżeli
         \[
             \left.\frac{d(\varphi_0\cdot \sigma_1(t))}{d t}\right|_{t=0} = \left.\frac{d(\varphi_0 \cdot \sigma_2(t))}{dt}\right|_{t=0}
        .\]
    \end{definicja}
    Rozważmy wszystkie krzywe przechodzące przez punkt $P\in M$. Na tym zbiorze wprowadzamy relację: $\sigma_1 \sim \sigma_2$ jeżeli $\sigma_1$ i $\sigma_2$ są styczne. Jeżeli $\sigma$ krzywa przechodząca przez punkt $P$, to wektorem stycznym zaczepionym w punkcie $P$ nazwiemy $v = \underset{\substack{\text{klasa}\\ \text{równoważności}}}{[\sigma]} $

    \begin{przyklad}
        Weźmy krzywą $\sigma(t) = \begin{bmatrix} \cos(t) \\ \sin(t) \\ t \end{bmatrix} , p = \begin{bmatrix} 1\\0\\0 \end{bmatrix} $.\\
        \[
            \sigma'(t) = \begin{bmatrix} -\sin(t)\\ \cos(t)\\ 1 \end{bmatrix}, \sigma'(0) = \begin{bmatrix} 0\\1\\1 \end{bmatrix}
        .\]

    \end{przyklad}

    \begin{figure}[h]
        \centering
        \incfig{fig_52}
        \label{fig:fig_52}
    \end{figure}

    \begin{przyklad}
        Niech $f(p) = C \underset{p\in M}{\forall} $. Ile wynosi $v(f)?$
         \begin{align*}
            &v(f) = v(c) = v(c \cdot 1) = c \cdot v(1) = \\
            &=  c \cdot v(1\cdot 1) = c \cdot (1 \cdot v(1) + 1v(1)) =\\
            &= c \cdot 2v(1) = 2v(c) = 2v(f)
        .\end{align*}
        Czyli $v(f) = 2v(f)$, czyli $v(f) = 0$ (pochodna stałej $=0$ )
    \end{przyklad}
    Każdy operator, który to umie to różniczkowanie.
    \begin{pytanie}
        Jak można w praktyce zrealizować taki operator?\\
        Niech $v\in T_pM, v=[\sigma]$
         \[
             v(f) = \frac{d}{dt}f(\sigma(t))|_{t=0}
        .\]
    \end{pytanie}
    \begin{definicja}
        Zbiór wszystkich różniczkowań w punkcie $P$ oznaczamy przez $D_pM$
    \end{definicja}
    Chcemy nadać $D_pM$ strukturę przestrzeni wektorowej.
    \begin{align*}
        &v_1,v_2\in D_pM, f\in \mathcal{C}^{\infty}(M) \implies (v_1 \diamond v_2)f \overset{\text{def}}{=} v_1(f) + v_2(f)\\
        &\underset{\alpha\in\mathbb{R}}{\forall} (\alpha \bowtie v_1)f = \alpha \cdot v_1(f) \\
    .\end{align*}
    \begin{pytanie}
        Co to znaczy, że $f$ - klasy $C^{\infty}(M)$?
    \end{pytanie}
    \begin{figure}[h]
        \centering
        \incfig{fig_53}
        \caption{$f$ nie musi być bijekcją jakby co}
        \label{fig:fig_53}
    \end{figure}

    Jeżeli $\psi \circ f \circ \varphi^{-1}$ - jest klasy $\mathcal{C}^{\infty}$.\\
    Związek między $T_pM$, a $D_pM$ :\\
    Niech $v = 5e_x + 6e_y \in T_pM$. Czy znajdziemy odwzorowanie z  $T_pM$ do $D_pM$, (które dokładnie jednemu $v$ przyporządkowałoby jeden element).$\rightarrow$ izomorfizm między $T_pM$ i $D_pM$.

    \subsection{asdasdasd}
    Zbiór wszystkich wektorów stycznych zaczepionych w punkcie $p\in M$ oznaczamy przez  $T_pM$ i nazywamy przestrzenią styczną. (Uwaga: warunek (*) nie zależy od wyboru mapy).

    Chcemy wyposażyć $T_pM$ w strukturę przestrzeni wektorowej. Potrzebujemy działań.\\
    Niech $v_1,v_2\in T_pM$ i $v_1 = [\sigma_1], v_2 = [\sigma_2]$. Wówczas\\

    \begin{align*}
         &v_1 \diamond v_2 \overset{\text{def}}{=} \left[ \varphi^{-1}(\varphi(\sigma_1))+\varphi(\sigma_2) \right]\\
         &\underset{\alpha\in\mathbb{R}}{\forall} \alpha\cdot v_1 \overset{\text{def}}{=} \left[ \varphi^{-1}(\alpha\cdot \varphi(\sigma_1) \right]
    .\end{align*}
    $T_pM$ wraz z działaniami ($\diamond, \cdot $ ) ma strukturę przestrzeni wektorowej. Zbiór
    \[
        TM \overset{\text{def}}{=} \left\{ p\in M, T_pM \right\}
    \]  nazywamy wiązką styczną.

    \begin{pytanie}
        Czy w $TM$ możemy zadać strukturę przestrzeni wektorowej?\\
        Odpowiedź: NIE DA SIĘ
    \end{pytanie}

    \subsection{Przestrzeń różniczkowa}
    Niech $f: M\to \mathbb{R}$, $f$ - klasy $\mathcal{C}^{\infty}(M)$ \\
    niech $v\left(  \right) : \mathcal{C}^{\infty}(M)\to\mathbb{R}$, takie, że
    \begin{align*}
    &\underset{f,g\in\mathcal{C}^{\infty}\left( M \right) }{\forall}v(f\cdot g) = v(f) + v(g)\\
    &\underset{\alpha\in\mathbb{R}}{\forall} \underset{f\in\mathcal{C}^\infty(M)}{\forall} v(\alpha f) = \alpha v(f)\\
    & \underset{f,g\in\mathcal{C}^\infty (M)}{\forall} v(f\cdot g) = f(p) \cdot v(g) + g(p)v(f)
    .\end{align*}
    $v\left(  \right) $ spełniający te warunki nazywamy różniczkowaniem w punkcie $p$.


\end{document}
