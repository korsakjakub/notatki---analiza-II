\documentclass[../main.tex]{subfiles}
\graphicspath{
    {"../img/"}
    {"img/"}
}

\begin{document}
\begin{przyklad}
    Zastanówmy się jak wygląda rotacja wektora w układzie sferycznym. $M = \mathbb{R}^3$.
    \[
        v \overset{\sharp}{\to} \Lambda^1(M) \overset{d}{\to} \Lambda^2(M) \overset{\ast}{\to} \Lambda^1(M) \to \overset{\flat}{T_pM} \to \begin{bmatrix} \\ \\ \end{bmatrix}_{i}
    \]
        $rot v = \left( \ast (d v^\sharp) \right)^{\flat}$\\
        na początek dostajemy w smsie
        \[
            \begin{bmatrix} A^r \\ A^\theta \\ A^\varphi \end{bmatrix}_{i_r, i_\theta, i_\varphi} = v = A^r \frac{\partial }{\partial r} + A^\theta \frac{1}{r} \frac{\partial }{\partial \theta} + A^\varphi \frac{1}{r\sin\theta} \frac{\partial }{\partial \varphi}
        \]
        chcemy sobie zrobić jednoformę, która jest podniesionym wektorkiem:
        \[
            \alpha = v^\sharp = g_{rr} A^r dr + g_{\theta\theta} \frac{1}{r} A^\theta d\theta + g_{\varphi\varphi} \frac{1}{r\sin\theta}A^\varphi d\varphi = A^r dr + r A^\theta d\theta + r\sin\theta A^\varphi d\varphi
        \]

        \begin{align*}
            d\alpha &= \left( A^r_{,\theta} - (rA^\theta)_{,r}  \right) d\theta \land dr + \left( A^r_{,\varphi} - (r\sin\theta A^\varphi)_{,r} \right) d\varphi \land dr +\\
            &+ \left( (rA^\theta)_{,\varphi} - (r\sin\theta A^\varphi)_{,\theta} \right) d\varphi \land d\theta\\
            \ast(dr\land d\theta) &= \sin\theta d\varphi,\quad \ast(d\theta\land d\varphi) = \frac{1}{r^2}dr,\quad \ast(d\varphi\land dr) = \frac{1}{\sin\theta}d\theta\\
            \ast d\alpha &= \left( (r \sin\theta A^\varphi)_{,\theta} - (rA^\theta)_{,\varphi} \right)\frac{1}{r^2\sin\theta} dr + \left( A^r_{,\varphi} - (r\sin\theta A^\varphi)_{,r} \right) \frac{1}{\sin\theta} d\theta + \\
        & + \left( (rA^\theta)_{,r} - A^r_{,\theta} \right) \sin\theta d\varphi
    .\end{align*}
    notacja: $\Box_{,\heartsuit} = \frac{\partial \Box}{\partial \heartsuit}$. Zostały nam jeszcze tylko dwie operacje.
    \begin{align*}
        &\left( \ast d\alpha \right) ^\flat = \left( (r\sin\theta A^\varphi)_{,\theta} - (r A^\theta)_{,\varphi}\right) \cdot 1\cdot \frac{1}{r^2\sin\theta} \frac{\partial }{\partial r} + \left( A^r_{,\varphi} - (r \sin\theta A^\varphi)_{,r}\right) \frac{1}{\sin\theta} \frac{1}{r^2} \frac{\partial }{\partial \theta} +\\
        &+ \left( (r A^\theta)_{,r} - A^r_{,\theta})\sin\theta \frac{1}{r^2\sin^2\theta} \right) \frac{\partial }{\partial \varphi}
    .\end{align*}
    Czyli
    \[
    rot \begin{bmatrix} A^r \\ A^\theta \\ A^\varphi \end{bmatrix} =
    \begin{bmatrix} \frac{1}{r^2\sin\theta} \left( (r\sin\theta A^\varphi)_{,\theta} - (r A^\theta)_{,\varphi}) \right) \\
    \frac{1}{r\sin\theta} \left( A^r_{,\varphi} - (r\sin\theta A^\varphi)_{,r} \right) \\
\frac{1}{r} \left( (r A^\theta)_{,r} - A^r_{,\theta} \right) \end{bmatrix}
    .\]
\end{przyklad}
\begin{przyklad}
    To może teraz dywergencja rzutem na taśmę.
    \begin{align*}
        &\begin{bmatrix} \\ \\  \end{bmatrix} = v \overset{\sharp}{\to} \Lambda^1(M) \overset{\ast}{\to} \Lambda^2(M) \overset{d}{\to} \Lambda^3(M) \overset{\flat}{\to} \Lambda^0 (M)  \\
        &div(v) = \ast\left( d(\ast v^\sharp) \right)\\
        &\begin{bmatrix} A^r\\ A^\theta\\ A^\varphi \end{bmatrix} = v, \alpha = v^\sharp\\
        &\alpha = A^r dr + r A^\theta d\theta + A^\varphi r \sin\theta d\varphi\\
        &\ast dr = r^2\sin\theta d\theta\land d\varphi\\
        &\ast d\theta = \sin\theta d\varphi\land dr\\
        &\ast d\varphi = \frac{1}{\sin\theta} dr\land d\theta\\
        &\ast \alpha = \left( A^r r^2 \sin\theta \right) d\theta \land d\varphi + \left( r \sin\theta A^\theta \right) d\varphi\land dr + \left( r A^\varphi \right) dr\land d\theta\\
        &d(\ast \alpha) = \left( (A^r r^2 \sin\theta)_{,r} + (r\sin\theta A^\theta)_{,\theta} + (r A^\varphi)_{,\varphi} \right) dr\land d\theta \land d\varphi\\
    .\end{align*}
    \[
        div \begin{bmatrix} A^r\\ A^\theta \\ A^\varphi \end{bmatrix} = \frac{1}{r^2\sin\theta} \left( (A^r r^2\sin\theta)_{,r} + (r\sin\theta A^\theta)_{,\theta} + (rA^\varphi)_{,\varphi} \right)
    .\]
    \begin{align*}
        &f(r,\theta,\varphi) \overset{d}{\to} \Lambda^1(M) \overset{\ast}{\to} \Lambda^2(M) \overset{d}{\to} \Lambda^3(M) \overset{\ast}{\to} \Lambda^0(M)\\
        &\alpha = df = \frac{\partial f}{\partial r} dr + \frac{\partial f}{\partial \theta} d\theta + \frac{\partial f}{\partial \varphi} d\varphi\\
        &\ast \alpha = \left( \frac{\partial f}{\partial r} r^2\sin\theta \right) d\theta\land d\varphi + \left( \frac{\partial f}{\partial \theta} \sin\theta \right) d\varphi\land dr + \left( \frac{\partial f}{\partial \varphi} \frac{1}{\sin\theta} \right) dr\land d\theta\\
        &d(\ast \alpha) = \left( \left(r^2\sin\theta \frac{\partial f}{\partial r} \right)_{,r} +\sin\theta \frac{\partial f}{\partial \theta} _{,\theta}  + \left( \frac{1}{\sin\theta} \frac{\partial f}{\partial \varphi} _{,\varphi} \right)  \right) dr\land d\theta\land d\varphi\\
        & \ast(d(\ast \alpha)) = \frac{1}{r^2\sin\theta}\left( \left(r^2\sin\theta \frac{\partial f}{\partial r} \right)_{,r} + \left(\sin\theta \frac{\partial f}{\partial \theta} \right)_{,\theta} + \left(\frac{1}{\sin\theta}\frac{\partial f}{\partial \varphi} \right)_{,\varphi} \right)
    .\end{align*}
\end{przyklad}
\begin{przyklad}
    $M = \mathbb{R}^3, f\in \Lambda^0(M)$.
    \begin{align*}
     &dd f = 0\\
     &ddf = d\left( \left( (df)^\flat \right)^\sharp \right) \implies rot( grad (f) ) = 0
    .\end{align*}
    Niech teraz $v\in \Lambda^1(M)$.
     \begin{align*}
         &d\left(\ast\left(\left(\ast(d V^\sharp)\right)^\flat\right)^\sharp\right) = \overset{\ast\ast=\mathbb{I}?}{d(\ast(\ast(d(v^\sharp))))} = dd(v^\sharp) = 0\\
         &div(rot (V)) = 0
    .\end{align*}
\end{przyklad}
Weźmy sobie jakąś funkcję: $f: (t,x,y,z)\to \mathbb{R}$.\\
        Zobaczmy jak $\ast d(\ast d f)$ wygląda w  $\begin{bmatrix} -1&&&\\&1&&\\&&1&\\&&&1 \end{bmatrix} $.
        \begin{align*}
            &df = \frac{\partial f}{\partial t} dt + \frac{\partial f}{\partial x} dx + \frac{\partial f}{\partial y} dy + \frac{\partial f}{\partial z} dz.\\
            &\ast\left( dx^{i_1}\land \ldots \land dx^{i_L} \right) = \frac{\sqrt{g} }{(n-L)!} g^{i_1j_1}\ldots g^{i_Lj_L}\in_{j_1\ldots j_kk_1\ldots k_{n-L}}dx^{k_1}\land\ldots\land dx^{k_{n-L}}\\
            & \ast(dx^0) = \frac{\sqrt{-(-1)} }{(4-1)!}g^{0 0}\in_{0k_1k_2k_3}dx^{k_1}\land dx^{k_2}\land dx^{k_3}, i,k = 0,\ldots,3\\
            &\ast(dx^0) = -\frac{1}{3!} 3! dx^1\land dx^2\land dx^3\\
            &\ast(dt) = -dx\land dy\land dz\\
            &\ast(dx^1) = \frac{\sqrt{-(-1)} }{(4-1)!} g^{1 1}\in_{1k_1k_2k_3}dx^{k_1}\land dx^{k_2}\land dx^{k_3}\\
            &\ast(dx) = 3! \frac{1}{3!}dy\land dt\land dz\\
            &\ast(dy) = dt\land dx\land dz\\
            &\ast(dz) = dx\land dt\land dy\\
            &\ast df = -\frac{\partial f}{\partial t} dx\land dy\land dz + \frac{\partial f}{\partial x} dy\land dt\land dz + \frac{\partial f}{\partial y} dt\land dx\land dz + \frac{\partial f}{\partial z} dx\land dt\land dy\\
            &d\ast d f =\left( -\frac{\partial ^2f}{\partial t^2} + \frac{\partial ^2f}{\partial x^2} + \frac{\partial ^2f}{\partial y^2} + \frac{\partial ^2f}{\partial z^2}   \right)dt\land dx\land dy\land dz
        .\end{align*}
        Na koniec:\\
        Mamy dwuformę pola elektromagnetycznego:
        \begin{align*}
            &F = -E_x dt\land dx + E_y dt\land dy - E_2 dt\land dz + B_x dy\land dz + B_y dz\land dy + B_z dy\land dx.\\
            &dF = 0 \text{ to jest pierwsza część równań Maxwella}\\
            &\begin{bmatrix} \rho\\ \rho v^x\\ \rho v^y\\ \rho v^z \end{bmatrix} = \begin{bmatrix} \rho\\ j^x\\ j^y\\ j^z \end{bmatrix}\\
            &j=-gdt + j^xdx+j^ydy+j^zdz\\
            &d(\ast F) = \ast j \text{ a to druga}
        .\end{align*}
\end{document}
