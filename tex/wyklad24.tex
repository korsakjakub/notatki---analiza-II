\documentclass[../main.tex]{subfiles}
\graphicspath{
    {"../img/"}
    {"img/"}
}

\begin{document}
    \begin{przyklad}
        (na pchnięcie wektora)\\
        Niech $M = \mathbb{R}^1, N=\mathbb{R}^3, h(t) = \begin{bmatrix} f(t)\\g(t)\\r(t) \end{bmatrix} $\\
        Niech $p\in \mathbb{R}^1$, niech $v\in T_pM, v = a \frac{\partial }{\partial t} $. $v = [\sigma], \tilde \sigma(t) = at+p$, $\sigma(c) = p, \frac{d \tilde \sigma(t)}{dt}\vert_{t=0} = a$.\\
        \[
            h_x \sigma = \begin{bmatrix} f(at+p)\\g(at+p)\\r(at+p) \end{bmatrix}, h_x v = [h_x \sigma], \frac{d}{dt}(\tilde h_x \sigma)\vert_{t=0}
        .\]
        \[
            h_xv = \begin{bmatrix} af'(p)\\ ag'(p) \\ ar'(p) \end{bmatrix} = af'(p) \frac{\partial }{\partial x} + ag'(p) \frac{\partial }{\partial y} + ar'(p) \frac{\partial }{\partial z}
        .\]
    \end{przyklad}
    \begin{figure}[h]
        \centering
        \incfig{fig_57}
        \label{fig:fig_57}
    \end{figure}
    \begin{definicja}
    Niech $M, N$ - rozmaitości, $h: M\to N$ i niech $p\in M, \alpha\in T^*_{h(p)}N$.\\
    Cofnięciem formy $\alpha$ w odwzorowaniu $h$ nazywamy formę $h^*\alpha \in T_pM$, taką, że $\left<h^*\alpha,v \right> = \left<\alpha,hv \right>\underset{v\in T_pM}{\forall} $ i caaa. Jeżeli $\alpha_1,\alpha_2,\ldots,\alpha_k \in \Lambda^1(N)$ i $v_1,\ldots,v_k\in T_p(M)$, to
    \[
        h^*(\alpha_1\land\ldots\land\alpha_k), v_1,\ldots,v_k \overset{\text{def}}{=} \begin{bmatrix} \left<h^*\alpha_1,v_1 \right>&\left<h^*\alpha_2,v_1 \right>&\ldots&\left<h^*\alpha_k,v_1 \right> \\
            \vdots &&&\\
            \left<h^*\alpha_k,v_k \right>&\left<h^*\alpha_k, v_k \right>&\ldots&\left<h^*\alpha_k,v_k \right>
        \end{bmatrix}
    .\]
    Czyli
    \[
        h^*(\alpha_1\land\ldots\land\alpha_k) = (h^*\alpha_1)\land(h^*\alpha_2)\land\ldots\land h^*(\alpha_k)
    .\]
    \end{definicja}
    \begin{figure}[h]
        \centering
        \incfig{fig_58}
        \caption{$\left<h^*\alpha,v \right> \overset{\text{def}}{=}  \left<\alpha, h_*v \right>$}
        \label{fig:fig_58}
    \end{figure}

    \begin{figure}[h]
        \centering
        \incfig{fig_59}
        \label{fig:fig_59}
    \end{figure}
    \begin{przyklad}
        (wstępny)\\
        Niech $\alpha = 3(x^2+y^2)dx - 2xdy + 2z^2dz, \alpha\in \Lambda^1(N)$ (jednoformy nad $N$, $\dim N = 3$, chociaż można dać więcej jak się chce).\\
        $h(t) = \begin{bmatrix} \sin(t)\\ \cos(t)\\ t \end{bmatrix} $. Czym jest $h^*\alpha$?\\
        \[
        \left<h^*\alpha,v \right> = \left<\alpha,h_xv \right>
        .\]
        Niech $v\in T_pM$ i  $v = a \frac{\partial }{\partial t} $. Zatem $h_x v = a \cos(p) \frac{\partial }{\partial x} - a \sin(p) \frac{\partial }{\partial y} + a\cdot 1 \frac{\partial }{\partial t} $.
        \begin{align*}
            &\left<\alpha,h_*v \right> = \left< 3 \left( \sin^2(t) + \cos^2(t)\right) dx - 2\left( \sin(t) \right) dy + 2\left( t^2 \right) dz, h_xv\right> =\\
            &= \left<3dx - 2\sin(t)dy + 2t'dz, a \cos(t) \frac{\partial }{\partial x} - a \sin(t) \frac{\partial }{\partial y} + a\cdot 1 \frac{\partial }{\partial z}  \right>_{t=p} \\
            &= 3a \cos(t) + 2a \sin^2(t) + at^2 \vert_{t=p} = \\
            &= \left< \left(3\cos(t)dt + 2a\sin^2(t) + at^2\right)\vert_{t=p}, a \frac{\partial }{\partial t}  \right> = \\
            & \text{czyli } h^*\alpha = \left( 3\cos(t) + 2 \sin^2(t) + t^2 \right) dt \\
        .\end{align*}
        Na skróty!\\
        \begin{align*}
            &x = \sin(t) &&dx = \cos(t)dt\\
            &y = \cos(t) &&dy = -\sin(t) dt\\
            &z = t &&dz = dt
        .\end{align*}
        Zatem
         \begin{align*}
             h^*\alpha &= 3\left( \sin^2(t) + \cos^2(t) \right) \cos(t)dt - 2\sin(t) \left( -\sin t dt \right) + 2t^2dt \\
                       &= \left( 3\cos(t) + 2\sin^2(t) + 2t^2 \right) dt
        .\end{align*}
    \end{przyklad}

    \begin{przyklad}
        Niech $M = \mathbb{R}^4, N = \mathbb{R}^4$.\\
        \begin{align*}
            &\gamma = \frac{1}{\sqrt{1-v^2} },\\
            &c = 1\\
            h:\quad &t = \gamma(t'-vx')\\
               &x = \gamma(x'-vt')\\
               &y = y'\\
               &z = z'
        .\end{align*}
        Czyli
        \begin{align*}
            &dt = \gamma(dt' - v dx')\\
            &dx = \gamma(dx' - v dt')\\
            &dy = dy'\\
            &dz = dz'
        .\end{align*}
        Chcemy cofnąć naszą formę. Na fizyce nie używamy słowa \textit{cofnięte}.
        \begin{align*}
            &F' = -E_x\left( \gamma\left( dt' - vdx' \right)  \right) \land \gamma \left( dx'-vdt' \right) - E_y \gamma\left( dt' - vdx' \right) \land dy'=\\
            &= -E_x \gamma^2 \left( 1 - v^2 \right) dt' \land dx' - E_y\gamma dt'\land dy' + E_y\gamma v dx'\land dy'= \\
            &= -E_x \frac{1}{1-v^2}\left( 1-v^2 \right) dt'\land dx' - E_y \gamma dt'\land dy' + \gamma v E_x dx'\land dy' \\
            &F' = -E'_x dt'\land dx' - E'_y dt' \land dy' + B'_z dx'\land dy' \\
        .\end{align*}
        Czyli
        \begin{align*}
            &E'_x = E_x\\
            &E'_y = \gamma E_y\\
            &B'_z = \gamma v E_y
        .\end{align*}
    \end{przyklad}

    \begin{figure}[h]
        \centering
        \incfig{fig_60}
        \label{fig:fig_60}
    \end{figure}

    Obserwacja: Niech $\alpha \in \Lambda^1(N)$, $\dim N = k$, niech $M$ - rozmaitość, $\dim M = n$ i $h: M\to N$. Wówczas
    \[
        h^*f\in \Lambda^0(M)
    .\]
    Oraz
    \[
        d(h^*f) = h^*(df)
    .\]

%   \begin{figure}[h]
%       \centering
%       \incfig{fig_61}
%       \label{fig:fig_61}
%   \end{figure}

    \begin{dowod}
        Skoro $f\in \Lambda^0(N)$, to $f(x^1,x^2,\ldots,x^k)$,\\
        $df = \frac{\partial f}{\partial x^1} dx^1 + \frac{\partial f}{\partial x^2} dx^2 + \ldots + \frac{\partial f}{\partial x^k} x^k$.
        \[
            \left<h^*(df),v \right> = \left<df,h_xv \right>, v\in T^pM
        .\]
        Niech $V\in T_pM$.\\
        \[
            \tilde h(t_1,\ldots,t_n) = \begin{bmatrix} h_1(t_1,\ldots,t_n)\\ \vdots \\ h_k(t_1,\ldots,t_n) \end{bmatrix}
        .\]
        Jeżeli $v = a_1 \frac{\partial }{\partial t^1} + a_2 \frac{\partial }{\partial t^2} + \ldots + a_n \frac{\partial }{\partial t_n} $, to $h_*v = \left( \begin{bmatrix} h' \end{bmatrix} \begin{bmatrix} a_1\\ \vdots \\ a_n \end{bmatrix}  \right)_{\frac{\partial }{\partial x^1} , \ldots, \frac{\partial }{\partial x^k} } $.\\
        \begin{align*}
            &h_xv = \left( \begin{bmatrix} \frac{\partial h_1}{\partial t^1} & \ldots & \frac{\partial h_k}{\partial t^k} \\ \vdots \\ \frac{\partial h_k}{\partial t^1} & \ldots & \frac{\partial h_k}{\partial t^k}  \end{bmatrix} \begin{bmatrix} a_1 \\ \vdots \\ a_n \end{bmatrix}  \right)_{\frac{\partial }{\partial x^1} ,\ldots, \frac{\partial }{\partial x^k} } = \left( \frac{\partial h_1}{\partial t^1} a_1 + \ldots + \frac{\partial h_1}{\partial t^n} a_n \right) \frac{\partial }{\partial x^1} +\\
&+ \ldots + \left( \frac{\partial h_k}{\partial t^1} a_1 + \ldots + \frac{\partial h_k}{\partial t^n} a_n \right) \frac{\partial }{\partial x^k}
        .\end{align*}
        Dalej
        \begin{align*}
            &\left<df,h_*v \right> = \frac{\partial f_1}{\partial x^1} \left( \frac{\partial h_1}{\partial t^1} a_1 + \ldots + \frac{\partial h_1}{\partial t^n} a_n \right)  + \ldots + \frac{\partial f}{\partial x^k} \left( \frac{\partial h_k}{\partial t^1} a_n + \ldots + \frac{\partial h_k}{\partial t^n} a_n \right) =\\
            &= \left<df(h_1(t_1,\ldots,t_n),h_2(t_1,\ldots,t_n),\ldots,h_k(t_1,\ldots,t_n)), a_1 \frac{\partial }{\partial t^1} + \ldots + a_n \frac{\partial }{\partial t^n}  \right> \\
        .\end{align*}
    \end{dowod}
%   \begin{figure}[h]
%       \centering
%       \incfig{fig_62}
%       \label{fig:fig_62}
%   \end{figure}

\end{document}
