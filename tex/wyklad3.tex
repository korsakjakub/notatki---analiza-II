\documentclass[../main.tex]{subfiles}
\graphicspath{
    {"../img/"}
    {"img/"}
}
\begin{document}

\begin{przyklad}

\textbf{Uwaga:} jeżeli np. $f: \mathbb{R}^{2} \to \mathbb{R}^{2}$, to znaczy, że\\
$f(x,y) =
\left [
\begin{matrix}
    f_1 (x,y)\\
    f_2 (x,y)
\end{matrix}
\right ], f_1 : \mathbb{R}^{2} \to \mathbb{R}^{1}, f_2 : \mathbb{R}^{2} \to \mathbb{R}^{1}$
, wówczas
\[
    \frac{\partial f}{\partial x} =
    \left [
        \begin{matrix}
            \frac{\partial}{\partial x} f_1 \\
            \frac{\partial}{\partial x} f_2
        \end{matrix}
    \right ],
    \frac{\partial}{\partial y} f =
    \left [
        \begin{matrix}
            \frac{\partial}{\partial y} f_1 \\
            \frac{\partial}{\partial y} f_2
        \end{matrix}
    \right ]
\]
\end{przyklad}

\begin{przyklad}
    \[
        f(x,y) = \left [
        \begin{matrix}
            2xy^2\\
            x^3 y
        \end{matrix}\right ]
    \]
Wtedy pochodne czątkowe:
\[
    \frac{\partial f}{\partial x} =\left [ \begin{matrix} 2y^2\\ 3x^2 y \end{matrix} \right ], \frac{\partial f}{\partial y} = \left [ \begin{matrix} 4xy\\ x^3 \end{matrix}\right ]
\]
\begin{align*}
        f(x+h)-f(x) &= \\
        &= \frac{\partial f}{\partial x} h^x + \frac{\partial f}{\partial y} h^y + r((x,y),h) = \\
        &=\left [ \begin{matrix} 2y^2\\ 3x^2 y\\ \end{matrix}\right ] h^x + \left [ \begin{matrix} 4xy\\ x^3\\ \end{matrix}\right ] h^y + r((x,y),h) \\
        &= \left [ \begin{matrix} 2y^2    &4xy\\ 3x^2 y  &x^3\\ \end{matrix}\right ] \left [ \begin{matrix} h^x\\ h^y\\ \end{matrix}\right ] + r((x,y),h)
.\end{align*}

Czyli
\[
        f' = \left [ \begin{matrix}
\frac{\partial f_1}{\partial x}     &\frac{\partial f_1}{\partial y}\\
\frac{\partial f_2}{\partial x}     &\frac{\partial f_2}{\partial y}\\
\end{matrix}\right ]
\]

i ogólniej: jeżeli $f: \mathbb{R}^n \to \mathbb{R}^k$, to
\[
        f' = \left [ \begin{matrix}
    \frac{\partial f_1}{\partial x^1}   &\dots   &\frac{\partial f_1}{\partial x^n}\\
    \vdots                              &\ddots  &\vdots\\
    \frac{\partial f_k}{\partial x^1}   &\dots   &\frac{\partial f_k}{\partial x^n}\\\end{matrix}\right ]
\]
\end{przyklad}

\subsection{
    Uzupełnienie:
}

\begin{stw}
    Niech $V$ - przestrzeń wektorowa z normą $||.||$ i $x_0\in V$, wówczas
    \[
        f(x)=||x||, f: V\to\mathbb{R}^1 \text{ - ciągła w } x_0.
    \]
\end{stw}

\begin{proof}

Chcemy pokazać, że
    \[
        \underset{\varepsilon > 0}{\forall} \quad\underset{\delta}{\exists} \quad\underset{x}{\forall} \quad d_x (x,x_0) < \delta \implies d_{\mathbb{R}} (f(x),f(x_0)) < \varepsilon
    \]
ale
    \[
        d_x(x,y) = ||x-y||, d_{\mathbb{R}^1} (x,y) = |x-y|
    .\]
Czyli pokażemy, że
    \[
        \underset{\varepsilon > 0}{\forall} \underset{\delta}{\exists} \underset{x}{\forall} \quad ||x - x_0|| < \delta \implies \big | ||x|| - ||x_0|| \big | < \varepsilon
    .\]
Ale wiemy, że
    \[
        ||x|| = ||x-y+y|| \leq ||x-y|| + ||y||, ||x||-||y||\leq ||x-y||,
    \]
    \[
        ||y|| = ||y-x+x||\leq ||y-x|| + ||x||,
    \]
    \[
        ||y||-||x||\leq ||x-y||
    ,\]
czyli $\left| \left\Vert x \right\Vert  - \left\Vert y \right\Vert  \right| \leq \left\Vert x-y \right\Vert $. Niech $\delta = \frac{\varepsilon}{2}$, otrzymujemy $\varepsilon > \frac{\varepsilon}{2} > ||x-y|| \geq \big | ||x|| - ||y|| \big | \geq 0$

\end{proof}

\begin{pytanie}
Niech $f(x,y) = 7x+6y^2 \text{ i } g(t) = \left [ \begin{matrix}
cos(t) \\
sin(t) \\
\end{matrix}\right ]$. Wówczas $h(t) = (f \circ g)(t) : \mathbb{R}\to \mathbb{R}$. Ile wynosi pochodna?
\end{pytanie}

$f' = [7,12y] , g' = \left [ \begin{matrix}
-sin(t)\\
cos(t)\\
\end{matrix}\right ]$

\begin{tw}
Niech $G:U \to Y, U\subset X, U$ - otwarte, \\
    $X$ - przestrzeń wektorowa unormowana, \\
    $F: G(U) \to Z, G(U) \subset V$\\
    $G$ - różniczkowalna w $x_0\in U$,\\
    $F$ - różniczkowalna w $G(x_0)\in U$.\\
Wówczas:
        $(F \circ G )$ - różniczkowalna w $x_0$ oraz
    \[
        (F \circ G)' (x_0) = \left . F'(x)\right |_{x=G(x_0)} G'(x_0).
    \]
\end{tw}


\begin{proof}
    \[
        G(x_0 + h_1) - G(x_0) = G'(x_0)h_1+r_1(x_0,h_1)\text{, gdy }\frac{r(x_0,h_1)}{||h_1||_x} \to 0
    \]
    \[
        F(y_0 + h_2) - F(y_0) = F'(y_0)h_2+r_2(y_0,h_2)\text{, gdy }\frac{r(y_0,h_2)}{||h_2||_y} \to 0
    \]

\begin{align*}
    F\left(G(x_0 + h)\right) - F(G(x_0)) &=\\
    &= F(G(x_0) + G'(x_0)h_1 + r_1(x_0,h_1)) - F(G(x_0)) =\\
    &= F(G(x_0)) + F'(G(x_0)) \cdot (G'(x_0)h_1 + r_1(x_0,h_1)) + \\
    &= r_2(G(x_0), G'(x_0)h_1 + r_1(x_0,h_1)) - F(G(x_0)).
\end{align*}
zatem:
\begin{align*}
    F(G(x_0+h)) - F(G(x_0)) &=\\
    &=F'(G(x_0)) \cdot G'(x_0)h_1+F'(G(x_0)) \cdot r_1(x_0,h_1)+\\
    &=r_2 \cdot (G(x_0),G'(x_0)h_1+r_1(x_0,h_1))
.\end{align*}

Wystarczy pokazać, że
    \[
        \frac{r_3}{||h_1||}\to 0,
    \]
    ale
    \begin{align*}
        \frac{r_3}{||h_1||} &= F'(G(x_0)) \frac{r_1(x_0,h_1)}{||h_1||} +\\
        &+ \underbrace{\frac{r_2(G(x_0),G'(x_0)h_1+r_1(x_0,h_1))}{||G'(x_0)h_1 + r_1(x_0,h_1)||}}_{\to 0 \text{ kiedy } h_1 \to 0} \cdot
        \underbrace{\frac{||G'(x_0)h_1+r_1(x_0,h_1)||}{||h_1||}}_{\text{jest ograniczony}}
    ,\end{align*}
    ale jeżeli $h_1\to 0$, to $h_2 = G'(x_0)h_1+r_1(x_0,h_1)$,
    zatem $F(G(x))$ - różniczkowalna w $x_0$
\end{proof}


\begin{przyklad}

$f(x,y) = \left [ \begin{matrix}
2xy^2\\
x^3 y\\\end{matrix}\right ],
\varphi(t) = \left [ \begin{matrix}
2t^2\\
t^3\\
\end{matrix}\right ],
h(t) = (f \circ \varphi) (t), h: \mathbb{R}\to \mathbb{R}^2$.

Policzmy $H'$. $f' = \left [ \begin{matrix}
2y^2    &4xy\\
3x^2y   &x^3\\
\end{matrix}\right ], \varphi '(t) = \left [ \begin{matrix}
4t\\
3t^2\\
    \end{matrix}\right ]$, tzn.
    \[
        H' = \left [ \left .\begin{matrix}
2y^2    &4xy\\
3x^2y   &x^3\\
    \end{matrix}\right ]\right |_{x=2t^2, y=t^3} \cdot \left [ \begin{matrix}
4t\\
3t^2\\
\end{matrix}\right ] = \\
\left [ \begin{matrix} 2(2t^2)^2 4t + 4(2t^2)(t^3) 3t^2\\
3(2t^2)^2 t^3 4 + (2t^3)^3 3t^2\\\end{matrix}\right ]
\]
\end{przyklad}

\begin{przyklad}
Niech $f: \mathbb{R}^2 \to \mathbb{R}$, \\
    $\Psi: \mathbb{R}^2 \to \mathbb{R}^2$,\\
    $\Psi(r,\varphi) = \left [ \begin{matrix} \Psi_1(r,\varphi)\\ \Psi_2(r,\varphi)\\\end{matrix}\right ]$
$\Psi_1: \mathbb{R}^2 \to \mathbb{R}$
$\Psi_2: \mathbb{R}^2 \to \mathbb{R}$

\vspace{0.3cm}
Niech $H(r,\varphi) = (f \circ \Psi) (r, \varphi)$, czyli $H : \mathbb{R}^2 \to \mathbb{R}$.

Szukamy pochodnej $H$, ale
    \[
        f' = [\frac{\partial f}{\partial x} , \frac{\partial f}{\partial y} ], \Psi ' = \left [ \begin{matrix}
\frac{\partial \Psi_1}{\partial r}  &\frac{\partial \Psi_1}{\partial \varphi} \\
\frac{\partial \Psi_2}{\partial r}  &\frac{\partial \Psi_2}{\partial \varphi}
        \end{matrix}\right ]
    \]

Czyli
    \[
        H' = \left . \Big [ \frac{\partial f}{\partial x} , \frac{\partial f}{\partial y} \Big ] \right |_{x=\Psi_1(r,\varphi), y=\Psi_1(r,\varphi)} \left [ \begin{matrix}
\frac{\partial \Psi_1}{\partial r}  &\frac{\partial \Psi_1}{\partial \varphi} \\
\frac{\partial \Psi_2}{\partial r}  &\frac{\partial \Psi_2}{\partial \varphi}
        \end{matrix}\right ]
    \]

Co daje:
    \[
        \left . \Big [ \frac{\partial H}{\partial r} , \frac{\partial H}{\partial \varphi} \Big ] = \Big [ \frac{\partial f}{\partial x} \frac{\partial \Psi_1}{\partial r} + \frac{\partial f}{\partial y} \frac{\partial \Psi_2}{\partial r} , \frac{\partial f}{\partial x} \frac{\partial \Psi_1}{\partial \varphi} + \frac{\partial f}{\partial y} \frac{\partial \Psi_2}{\partial \varphi} \Big ] \right |_{x = \Psi_1 (r,\varphi), y = \Psi_2 (r,\varphi)}
    \]
\end{przyklad}
\end{document}
