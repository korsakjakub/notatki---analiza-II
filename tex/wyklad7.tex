\documentclass[../main.tex]{subfiles}
\graphicspath{
    {"../img/"}
    {"img/"}
}
\begin{document}

\begin{tw}
    (o lokalnej odwracalności)\\
    Niech
    \begin{align*}
        &f: E \to E, E\text{ - otwarty, } E\subset \mathbb{R}^n,f \in \mathcal{C}^1 (E), \\
        &\underset{a,b\in E}{\exists}:f(a) = b \text{ i } f'(a) \text{ - odwracalna } (det(f'(a))\neq 0),
    \end{align*}
        to wtedy:
    \begin{align*}
        &\text{1. } \underset{U,V\subset E}{\exists}\quad \underset{a\in U, b\in V}{\exists}\quad U,V \text{ - otwarte, }f \text{ - bijekcja między } U,V\\
        &\text{2. } \underset{g: V\to U}{\exists}\quad \underset{x\in V}{\forall}\quad f(g(x)) = x,\\
        &\text{3. } g \in \mathcal{C}^1(V).
    \end{align*}
\end{tw}
\textbf{Uwaga:}
dowód składa się z trzech części:
\begin{itemize}
    \item Pokażemy, że $\underset{U,V}{\exists}: f \text{ - bijekcja na }U,V$
    \item Pokażemy, że $U,V$ - otwarte
    \item Pokażemy, że $\underset{g: V\to U}{\exists}, g$ - różniczkowalna na $V$ i ciągła.
\end{itemize}

\begin{przyklad}
$f(x,y) = \left [ \begin{matrix}
e^x \cos y\\
e^x \sin y\\ \end{matrix}\right ]
,
f'(x,y) = \left [ \begin{matrix}
e^x \cos y &-e^x \sin y\\
e^x \sin y &e^x \cos y\\
 \end{matrix}\right ] $

$det(f'(x,y)) = e^{2x} \neq 0$, ale $f(x,y) = f(x,y+2\pi)$
\begin{tiny}
(czyli funkcja jest okresowa)
\end{tiny}
\end{przyklad}


\begin{proof}
    \textbf{Część I}\\
Szukamy $U,V: f$ - bijekcja miedzy $U$ i $V$.\\
Skoro $f'(a)$ - odwracalne, to znaczy, że $\underset{(f'(a))^{-1}}{\exists}$, zatem
    \[
        \underset{\lambda}{\exists}: 2\lambda \Vert (f'(a))^{-1} \Vert = 1.
    \]

Wiemy, że $f'(x)$ - ciągła w $x=a$, czyli
    \[
    \underset{\varepsilon>0}{\forall}.\underset{\delta}{\exists}.\underset{x}{\forall},d(x,a)<\delta \implies \Vert f'(x) - f(a) \Vert < \varepsilon
\]
Połóżmy $\varepsilon = \lambda$.\\
Oznacza to, że
    \[
\underset{\delta_\lambda}{\exists}.\underset{x}{\forall}x\in K(a,\delta_\lambda) \implies \Vert f'(x) - f'(a) \Vert < \lambda
\]
Więc $U = K(a,\delta_\lambda)$, niech $V = f(U)$. Chcemy pokazać, że $f$ - bijekcja między $U$ i $V$.\\

Wprowadźmy funkcję pomocniczą:
    \[
\varphi_y(x) = x+[f'(a)]^{-1} (y-f(x)), x,y\in E
\]

\begin{pytanie}
    Co by było gdyby $\varphi_y(x)$ posiadała punkt stały? (jakie własności $x$ by z tego faktu wynikały)\\
dla $x\in U, y\in V, (y\in f(a))?$
\end{pytanie}

Z zasady Banacha wiemy, że odwzorowanie zwężające ma dokładnie jeden punkt stały, czyli
    \[
        \underset{y\in V}{\forall}\quad \underset{x\in U}{\exists}: f(x) = y
    \]

\textbf{Uwaga: } o $f$ - z taką własnością mówimy, że jest 1-1 na $U$.\\
    Policzmy $\varphi_y'(x)$
    \[
    \varphi_y '(x) = \mathbb{I} + (f'(a))^{-1} (-f'(x)) = (f'(a))^{-1}(f'(a) - f'(x)),
\]
    więc
    \begin{align*}
        &\Vert \varphi_y '(x) \Vert = \Vert f'(a)^{-1} (f'(a) - f'(x))\Vert \leq\\
        &\leq \Vert (f'(a)^{-1})\Vert \Vert f'(a) - f'(x) \Vert \leq\\
        &\leq \underset{x\in U}{\forall}\frac{1}{2\lambda} \lambda = \frac{1}{2}
    .\end{align*}

Pamiętamy, że jeżeli $\underset{M}{\exists}\Vert\varphi_y '(x)\Vert \leq M$, to $\underset{x,y}{\forall}\Vert \varphi(x) - \varphi(y)\Vert < M\Vert x-y \Vert$

Zatem skoro $\Vert \varphi_y '(x) \Vert \leq \frac{1}{2}$, to
$$\underset{x_1,x_2\in U}{\forall}\Vert \varphi_y(x_1) - \varphi_y (x_2) \Vert \leq \frac{1}{2} \Vert x_1 - x_2 \Vert,$$
więc $\varphi$ - zwężający na $U$, więc posiada dokładnie jeden punkt stały $\underset{y\in V}{\forall}$. Zatem $f$ - bijekcja między $U$ i $V$. $\Box$

    \vspace{0.5cm}
\textbf{Część II}\\
Zbiór $U$ - otwarty (bo tak go zdefiniowaliśmy) $U = K(a,\delta_1)$, więc
    \[
        \underset{x_0\in U}{\exists}\quad \underset{r}{\exists}K(x_0,r)\subset U
    \]
    lub równoważnie
    \[
        \Vert x-x_0 \Vert \leq r \land x\in U
    .\]
Chcemy pokazać, że dla $y_0 = f(x_0) \quad\exists\quad K(y_0,\lambda r)\subset V$, czyli że $V$ - otwarty.\\
\begin{figure}[h]
    \centering
    \includegraphics[width=0.8\textwidth]{fig_18}
    \caption{Trochę jak listy do św. Mikołaja (??)}
    \label{fig:fig_18}
\end{figure}
Weźmy $y\in K(y_0,\lambda r)$. Zauważmy, że $\varphi_{y_1}(x_1)$ - zwężające, jeżeli $y_1\in V, x_1\in U$\\
Jeżeli pokażemy, że dla $\Vert y-y_0\Vert < \lambda r, \varphi_y(x)$ - zwężająca na $K(x_0,r)\subset U$, to będziemy wiedzieli, że $\Vert y - y_0 \Vert < \lambda r$ oraz $y\in V \iff K(y_0,\lambda r)\subset V$

\begin{figure}[h]
    \centering
    \includegraphics[width=0.6\textwidth]{fig_19}
    \caption{Nie ok.}
    \label{fig:fig_19}
\end{figure}

Żeby pokazać, że $\varphi_y(x)$ - zwężające na $K(x_0,r)$, zbadamy tę wielkośc dla $x\in K(x_0,r)$.
$\Vert \varphi_y(x) - x_0 \Vert$, chcielibyśmy, aby $\Vert \varphi_y(x) - x_0 \Vert \leq r $ i $\Vert y - y_0 \Vert < \lambda r$, ale z drugiej strony
\[
    \Vert \varphi_y(x) - x_0 \Vert = \Vert \varphi_y(x) - \varphi_y(x_0) + \varphi_y(x_0) - x_0 \Vert \leq \Vert \varphi_y(x) - \varphi_y(x_0) \Vert + \Vert \varphi y(x_0 - x_0) \Vert.
\]

Ale
\[
    \Vert \varphi_y(x_0) - x_0 \Vert \leq \Vert (f'(a))^{-1} \Vert \Vert y - y_0 \Vert \leq \frac{1}{2\lambda} \lambda r = \frac{r}{2},
\]
więc
\[
    \Vert \varphi_y(x) - x_0 \Vert \leq r,
\]
jeżeli
\[
    \Vert y - y_0 \Vert < \lambda r, \Vert x - x_0 \Vert \leq r
.\]

Stąd wiemy , że punkt stały dla $\varphi_y(x):x\in K(x_0,r)$ należy do $K(x_0,r)$ i $\Vert y - y_0 \Vert <\lambda r$, zatem $y=f(x)$, czyli $V$ - otwarty.

\begin{figure}[h]
    \centering
    \includegraphics[width=0.5\textwidth]{fig_20}
    \caption{}
    \label{fig:fig_20}
\end{figure}

\textbf{Część III}\\
Szukamy $g: V\to U$\\
Skoro $f$ - bijekcja między $U$ i $V$, to znaczy, że $\underset{g: V\to U}{\exists}f(g(x)) = x \underset{x\in V}{\forall}$.\\
Chcemy pokazać, że $g(x)$ - różniczkowalne. Wiemy, że $f$ - różniczkowalna w $x\in U$, czyli
$$\frac{f(x+h) - f(x) - f'(x)h}{\Vert h \Vert} \overset{h\to 0}{\to} 0, x,x+h\in V$$
Jeżeli pokażemy, że
\begin{equation}
    \frac{g(y+k) - g(y) - [f'(x)]^{-1} k}{\Vert k \Vert} \overset{k\to 0}{\to} 0
\end{equation}
to będziemy wiedzieli, że:\\
1. $g$ - różniczkowalne dla $y\in V$\\
2. $g'(y)=[f'(x)]^{-1}$.\\
W tym celu pokażemy, że:\\
1. $(\Vert k \Vert \to 0) \implies (\Vert h \Vert \to 0)$\\
2. $[f'(x)]^{-1}$ istnieje dla $x\in U$. (na razie wiemy, że $(f'(a))^{-1}$ istnieje)

\begin{large}
    \textit{Ad 1.}
\end{large}
Zauważmy, że $$\varphi_y(x+h) - \varphi_y(x) = x+h+[f'(a)]^{-1}(y-f(x+h)) - x - [f'(a)]^{-1}(y-f(x))=$$
$$=h+[f'(a)]^{-1}(y-f(x+h)-y+f(x)) = h - (f'(a))^{-1}(f(x+h) - f(x)),$$
$$czyli \Vert \varphi_y(x+h) - \varphi_y(x) \Vert = \Vert h - (f'(a))^{-1}(k) \Vert \leq \frac{1}{2} \Vert h \Vert,$$
zatem $\Vert h - (f'(a))^{-1}k\Vert \leq \frac{1}{2}\Vert h \Vert \implies \Vert k \Vert \geq \Vert h \Vert, k = f(x+h) - f(x)$

Stąd ostatecznie mamy:
$\frac{g(y+k)-g(y) - [f'(x)]^{-1}k}{\Vert k \Vert} = [f'(x)]^{-1} \frac{hf'(x) - f(x+h) + f(x)}{\Vert k \Vert} \leq \frac{[f'(x)]^{-1}}{\lambda} \frac{hf'(x) - f(x+h) + f(x)}{\Vert h \Vert} \to 0$, o ile $\underset{[f'(x)]^{-1}}{\exists}$

\begin{pytanie}
    skąd wiadomo, że $(f'(x))^{-1}$?\\
\end{pytanie}
Wiemy, że $f'(a)$ jest odwracalna, więc $(f'(a))^{-1}$ istnieje, $a  \in U$.\\
Chcemy pokazać, że $f'(x)$ jest odwracalna dla $x\in U$. Oznacza to, że
\[
    0 < \Vert f'(x) y \Vert \text{ dla } y\neq 0, x\in U
.\]
Pamiętamy, że  $2\lambda \Vert (f'(a))^{-1} = 1$ oraz $U$ - taka, że
\[
    \underset{x\in U}{\forall} \Vert f'(x) - f'(a) \Vert < \lambda
.\]
Zatem \[
    0 \le \frac{1}{\Vert (f'(a))^{-1} \Vert } \Vert y \Vert = \Vert (f'(x) + f'(a) - f'(x))y \Vert \le \Vert f'(a) - f'(x) \Vert \Vert y \Vert + \Vert f'(x) \Vert \Vert y \Vert
.\]

Dalej $2\lambda \Vert y \Vert \le \lambda \Vert y \Vert + \Vert f'(x) y \Vert $ dla $x\in U$ \\
$0 \le \lambda \Vert y \Vert \le \Vert f'(x) y \Vert $ dla $y = 0$ \\
Czyli \[
    \underset{x\in U}{\forall} \Vert f'(x) y \Vert > 0
.\]
\end{proof}

\end{document}
